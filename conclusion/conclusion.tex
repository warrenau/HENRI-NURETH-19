\section{Conclusions} \label{s:conclusion}
The HENRI prototype built at OSU is intended to provide TREAT with greater capability to simulate RIA transients for LWRs.
Previous work has proved the feasibility of the system to meet the pressurization requirements of \SI{1.72}{\mega\pascal} (\SI{250}{psi}) in \SI{5}{\milli\second} \cite{HeNURETH}.
This work presents a solution for a valve to connect the high pressure driver tank to the low pressure test section.
A valve made from a pneumatic piston, operated with the system's helium, meets the physical constraints for the HENRI system and opens fast enough to provide the desired pressurization.
The piston also provides much greater consistency over rupture disks and it can be re-used over many tests.
The durability of a metal-to-metal seal was proven to be poor, but an o-ring design with a symmetrical mount has improved the low pressure sealing ability, however, the o-ring seal durability needs to be investigated and characterized further before design finalization.