\section{Experimental Facility and Methodology} \label{s:experiment}

The experiments for the piston valve were performed using much of the facility and procedure used for the rupture disk tests. This section will describe the experimental facility, highlighting the modifications from the rupture disk experiments, and the methodology used to test the piston valve.

% talk about basic methodology of experiments and testing, introduce the structure of the section


% section for experimental facility, comparison of rupture disk and piston valve test set ups
\subsection{Experimental Facility} \label{s:facility}

% P&ID, photos, etc. of facility -- highlight differences between burst disk and piston
[ADD stuff about physical constraints from TREAT?]The main facility, seen in \Cref{fig:HENRI Facility}, is set up similarly to a shock tube, with a driver tank on the floor, an intermediate tube section, a test section, and a reflector section. The whole facility is upside-down to the HENRI facility's orientation in the TREAT core. In the reactor, the driver tank will be above the core with the test section in the fuel. The reflector section will extend past the fuel to provide space for the helium-3 collision effects outside of the active region in the core. The intermediate section allows for the driver tank to be positioned above the reactor core while also decreasing the total required volume of helium-3 for the HENRI system. The facility is made using 304 stainless steel, with the test section and reflector being $1\frac{1}{2}$ inch schedule 40 NPS pipe; the intermediate section being $1\frac{1}{4}$ inch tubing with $0.125$ inch wall thickness; and the driver tank being 6 inch schedule 40 NPS pipe. [VOLUMES and LENGTHS?] The piston valve is attached to the top flange of the driver tank and hangs inside.

%
\begin{figure}[htbp]
    \vspace{16pt}
    \centering
    \includegraphics[width=3.024in, height=4.032in]{experiment/photos/Out_of_Pile_Experiment_Setup.jpg}
    \caption{HENRI Facility.}
    \label{fig:HENRI Facility}
    \vspace{16pt}
\end{figure}
%

The facility uses 13 pressure transducers and 1 thermocouple. The pressure transducers are one of two main types: Omega PX459 or PCB 113B24. All of the Omega pressure transducers are 0-1500 psia, except for two: a 0-3500 psia transducer used on the driver tank, and a 0-15 psia transducer used for the vacuum line. The Omega pressure transducers output a current from 4-\SI{20}{\milli\amp}, which is converted to voltage using a \SI{250}{\ohm} shunt resistor. The PCB transducers output a signal that is sent through a PCB signal conditioner before being output to the voltage card. The data acquisition system uses a National Instruments (NI) PXIe-1085 chassis with a NI [CONTROLLER \# here] controller, a NI 4303 voltage input card for reading the instrumentation, and a NI [NEED DO card \# here] for controlling the solenoid valves. The manifold of miniature solenoid valves used for controlling the piston valve can be seen in \Cref{fig:sv manifold}. These solenoid valves are Parker Series 9 Miniature Calibrant Valve 009-0172-900. This particular valve was selected for its small form factor, high pressure rating, and low leak rate. [TABLE?]

% %
% \begin{figure}[htbp]
%     \vspace{16pt}
%     \centering
%     \includegraphics[width=3.024in, height=4.032in]{experiment/photos/Piston_Assembly.jpg}
%     \caption{Piston assembled with second generation mount on flange.}
%     \label{fig:piston assembly}
%     \vspace{16pt}
% \end{figure}
% %


%
\begin{figure}[htbp]
    \vspace{16pt}
    \centering
    \includegraphics[width=4.032in, height=3.024in]{experiment/photos/SV_manifold.jpg}
    \caption{Miniature solenoid valve manifold.}
    \label{fig:sv manifold}
    \vspace{16pt}
\end{figure}
%




% section for detailed experimental method (test matrix)
\subsection{Experimental Methodology} \label{s:methodology}

% we tested the plugs and o-rings for durability / sealing with sustained seal tests and repeated seal tests outside of the henri facility (just using the flange and piston)
% we tested the valve / plugs / o-rings in henri tests

The overall methodology for testing the piston valve includes two main parts: testing the plug designs for durability and sealing after large numbers of cycles; and testing the piston valve's performance in HENRI helium insertions. Testing the plugs was performed by cycling the piston valve to simulate the wear of long term use and comparing its sealing ability before and after the cycles. The LabVIEW code used for data acquisition was modified to automate the piston cycling process, so the plugs could be cycled overnight, then tested the next day to determine the seal degradation due to wear. This method was used for both the metal-to-metal seal plug and the o-ring plug. Testing the valve performance was completed by performing helium insertions at varying initial pressures to characterize the valve at a wide range of pressures, as well as to provide data to compare the piston valve to the rupture disks. 